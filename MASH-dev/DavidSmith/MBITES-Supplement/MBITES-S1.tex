\documentclass{article}
\newcommand{\eg}{{\em e.g. }}
\usepackage{mathtools}
\usepackage{Sweave}
\begin{document}
\Sconcordance{concordance:MBITES-S1.tex:MBITES-S1.Rnw:%
1 4 1 1 0 330 1}


\section{Mosquito Bouts}

\subsection{Blood Feeding Search Bout [F]}

If a female mosquito is not in the neighborhood of a blood meal host, it  must search for one. She might also initiate a search for blood, even if she is at a suitable place to feed. In some cases, such as peri-domestic breeding, where it is not necessary to move to find a bloodmeal host, the mosquito might bypass a blood meal search flight.

A flight search bout for blood feeding could begin anywhere, but it always ends at a {\em feeding site} or {\em haunt}:
%
\begin{equation}
%
\left\{  l \right\}  
\cup 
\left\{  f \right\}  \cup \ldots 
\overset{{\cal K}_F}{\rightarrow} 
\left\{  f \right\} 
%
\end{equation}
%
The outcome of a search bout is either mosquito death, a new (possibly the same) location and behavioral state, or a frustrated search attempt without a change of behavioral state.

\paragraph{Implementation}

\begin{itemize}
%
\item A parameter, \verb1bfsb.s1, determines whether a mosquito has a ``success'' on its approach to the haunt. If the flight was a ''success,'' then its behavioral state is set to {\bf B}. Otherwise, it remains in state {\bf F}. 
%
\item The function  \verb1restingSpot()1 is then called. A mosquito can fail to find a suitable resting place and thus decided to leave (denoted \verb1bfrs.f1) in which case its behavioral state is set to {\bf F}. Otherwise, it's behavioral state is not changed. 
%
\item The functions \verb1flightStress()1 is then called, which computes the probability of surviving the flight, \verb1bfsb.fs1. If a mosquito dies, its state is set to {\bf D}.
%
\item The functions \verb1restingHazards()1 is then called, which computes the probability of surviving from landing through launch, \verb1bfsb.rs1. If a mosquito dies, its state is set to {\bf D}.
%
\end{itemize}

The mosquito's ending state is: 

\begin{itemize}
\item[{\bf D}] with probabilty $P_{FD} = 1 - bfsb.fs \cdot bfsb.rs $.  
\item[{\bf F}] with probability $P_{FF} = (1-P_{FD}) (bfrs.f + (1-bfrs.f)(1-bfsb.s))$.
\item[{\bf B}] with probability $P_{FB} = 1-P_{FF}$. 
\end{itemize}

\subsection{Blood Feeding Attempt Bout [B]}

A blood feeding bout begins when a mosquito chooses a host to approach. Information about the human hosts present in each haunt are stored in a queue, called {\em atRisk}. Other potential blood meal hosts (\eg cattle) can also be present. Each host has a weight, which is used in a multinomial probability distribution function to select one of the hosts to approach. 

After selecting a host, different functions determine the outcome of an attempt, depending on whether the host is human or another vertebrate animal. As a mosquito approaches a human attempting to blood feed, there are several steps: an approach, probing, and the blood meal. The mosquito could be frustrated or killed at each one of these steps (\eg from swatting). A mosquito blood feeding on a human becomes infected with a pathogen with some probability. Pathogens mature by a (possibly temperature dependent) rule determining EIP at some point in the future, if the mosquito is still alive. 

The process for feeding on a non-human vertebrate animals is simpler; parameters determine the probability of death or success (and their complement, failure followed by a new attempt). 

After successfully blood feeding, a mosquito's behavioral state changes to the post-prandial resting bout. If a mosquito survives the attempt but does not succeed in taking a blood meal, then it can either repeat a blood feeding attempt or leave the area to search for a new site.  

If a mosquito has successfully blood fed, a random variate is drawn to determine the size of the blood meal, and there is another function that determines whether the mosquito survives the stress associated with blood feeding, depending on the size of the blood meal.

\paragraph{Post Prandial Resting Bout [R]}

The outcome of a post prandial resting bout is either death, another attempt to blood feed, or an egg laying search bout.

Transition to egg laying behaviors depends on a rule determining how eggs mature, depending on the time required for egg maturation {\em vs.} the time required for a feeding bout. If egg maturation occurs every feeding bout, then blood is provisioned into eggs; the larger the blood meal, the larger the egg batch. A mosquito could choose to feed again, depending on the size of the blood meal / egg batch. Another option is that the eggs require time or blood resources to mature. If eggs are maturing, a mosquito repeats a blood feeding attempt until the eggs are mature.

If a mosquito survives, but the eggs are not mature or if she decides to top-up, she will attempt another blood meal. Even if eggs are mature, the mosquito may attempt to blood feed again, depending on the size of the last bloodmeal. Otherwise, a surviving mosquito makes an attempt to lays eggs. 

\paragraph{Egg Laying Search Bout [L]}

A female mosquito must search for an egg laying habitat, though (as in the case of peri-domestic breeding) one might be readily available. In such cases, the ``search'' results only in a change of state. A flight search bout for blood feeding could begin anywhere, but it always ends at a haunt:
%
\begin{equation}
%
\left\{ l\right\}  \cup \left\{ f\right\}  \overset{L}{\rightarrow} \left\{ l\right\} 
%
\end{equation}
%
The outcome of a search bout is either mosquito death, success and a change of location and behavioral state, or a frustrated search attempt with a change of location but without changing behavioral state. In some cases, such as peri-domestic breeding, where a mosquito finds herself already in an egg laying site, the mosquito might bypass an egg laying search flight. 

\paragraph{Egg Laying Attempt Bout [O]}

An egg laying bout involves a series of egg laying attempts where a mosquito dies, successfully lays some fraction of the eggs and initiates a new egg laying search bout, or is frustrated. If frustrated, she will either either make another attempt to lay eggs or initiate a new attempt somewhere else. 


\paragraph{Maturation}

\paragraph{Male Mosquito Populations}

\paragraph{Mating Bouts and Maturation}

Mating requires simulating male mosquito populations, and MBITES can simulate both female and male individuals. After an adult female has emerged from her pupal case and hardened, she is considered to be {\em pre-gonotrophic}. The conditions for maturation include a mating requirement and (possibly) an energetic requirement, meaning that she could not lay eggs until she had mated and satisfied an early life energy intake requirement. The model has not pre-determined what these pre-gonotrophic energetic requirements are, but provides a flexible set of functions and parameters to stipulate, depending on the species and either evidence or expert opinion. Mating and maturation can also be turned off, such that a female is mature and mated upon emergence.  

Male mosquitoes emerge from aquatic habitats and by default attempt to sugar feed (see below). At a specific time of each day, the male behavioral state switches to {\bf M}, and it moves to a mating point, $\left\{ m\right\} $, to be present for any female arriving at that point to mate. Male search thus always begins at a mating or sugar feeding site, $\left\{ s\right\} $ to arrive at a mating site $\left\{ m\right\} $:
%
\begin{equation}
%
\left\{ s\right\} \cup \left\{ m\right\}  \overset{{\cal K}_M}{\rightarrow} \left\{ m\right\} 
%
\end{equation}
%
If the male mosquito survives the flight, it enters a swarm. The {\em swarming queue} tracks all the male mosquitoes present at each point in $\left\{ m\right\} $ on each day. After the swarm disperses, surviving males behavioral states switch back to {\bf S} to sugar feed (see below). 

A pre-gonotrophic female does not lay eggs, though she can blood feed. Female mating bouts in this model occur only when she is in a pre-gonotrophic state, which (like males) can be triggered during a specific time window each day, when her state switches to {\bf M}. The female search kernel could thus begins at any mating, blood feeding, or sugar feeding site but always ends at a mating site $\left\{ m\right\} $:
%
\begin{equation}
%
\left\{ f\right\} \cup \left\{ s\right\} \cup \left\{ m\right\}  \overset{{\cal K}_M}{\rightarrow} \left\{ m\right\} 
%
\end{equation}
%
A female chooses from among the males in the queue at that location. The probability of success is a function of the number of males present in the swarm. When a female mosquito mates, the identity of the male mosquitoes she mated with are chosen from those present at the swarming site and stored. The outcome of a mating bout is either death, success, or failure. After a mating bout, a female's behavioral state switches to search for a blood meal, {\bf F}. 

An alternative to swarming behavior is opportunistic mating during some other activity (\eg blood feeding). If males are present where females blood feed, a random variate is drawn to see if the blood feeding attempt bout is interrupted and mating occurs. In such cases, ordinary hazards apply, and if the female survives, her behavioral state does not change. (NOTE: we need to add this possibility, just as we do for sugar feeding.)  

\paragraph{Sugar Feeding Bouts and Energetics}

A variable track's the level of each mosquito's energy reserves, and each bout depletes a mosquito's energy reserves by some amount. These energy reserves could be restored by blood feeding, depending on the species (a parameter is set to determine how much energy a female mosquito is able to derive from blood). Otherwise, a mosquito must restore its energy reserves by sugar feeding. Sugar sources are distributed across the landscape, at the points in $\left\{ s\right\} $, including possibly at swarming sites, haunts, and habitats. Sugar feeding could occur in two ways, through a sugar feeding bout or opportunistically.

A female mosquito's energy reserves are checked at the end of every other kind of bout, and a function determines whether a mosquito switches from some other state into an active sugar feeding behavioral state, thereby initiating a sugar feeding bout. A mosquito sugar feeding bout could begin at any haunt or at another habitat, but it always ends in a sugar feeding site:
%
\begin{equation}
%
\left\{ f\right\} \cup \left\{ l\right\}  \cup \left\{ m\right\}  \cup \left\{ s\right\}  \overset{L}{\rightarrow} \left\{ s\right\} 
%
\end{equation}
%
The outcome of a sugar bout is either death, a sugar meal, or another attempt. If a sugar feeding bout succeeds, the sugar reserves are topped up. 

During any other kind of bout, opportunistic sugar feeding can be triggered in a female if sugar is present. The functions triggering obligate and opportunistic feeding are, by default, shifted so that a female mosquito is more likely to feed opportunistically than to actively sugar feed. The sugar reserves of a mosquito are checked. The sugar at the source has a ``search weight,'' and functions determine whether a mosquito chooses to feed based on energy reserve levels and the weight. Another function determines how much the reserves are topped up. If a mosquito does sugar feed opportunistically, the mosquito could die during the attempt. If it survives opportunistic feeding, its behavioral state remains the same. 

Male mosquitoes are obligate sugar feeders when they are not mating. Sugar feeding will occur repeatedly until the time window when mating occurs, when its state shifts back to mating. 



\paragraph{Estivation [E]}

Estivation is a state of inactivity. In this model, estivation is induced seasonally, whenever certain conditions are met (\eg no rain). If a mosquito survives estivation, it initiates a blood feeding search bout at some point in the future.

\subsection{Options}

\paragraph{Resting Spot}

\paragraph{Senescence}

\paragraph{Wing Tattering}






\end{document}
